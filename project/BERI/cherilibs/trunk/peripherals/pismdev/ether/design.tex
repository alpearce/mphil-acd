Copyright (c) 2011 Wojciech A. Koszek
All rights reserved.

This software was developed by SRI International and the University of
Cambridge Computer Laboratory under DARPA/AFRL contract FA8750-10-C-0237
("CTSRD"), as part of the DARPA CRASH research programme.

@BERI_LICENSE_HEADER_START@

Licensed to BERI Open Systems C.I.C. (BERI) under one or more contributor
license agreements.  See the NOTICE file distributed with this work for
additional information regarding copyright ownership.  BERI licenses this
file to you under the BERI Hardware-Software License, Version 1.0 (the
"License"); you may not use this file except in compliance with the
License.  You may obtain a copy of the License at:

  http://www.beri-open-systems.org/legal/license-1-0.txt

Unless required by applicable law or agreed to in writing, Work distributed
under the License is distributed on an "AS IS" BASIS, WITHOUT WARRANTIES OR
CONDITIONS OF ANY KIND, either express or implied.  See the License for the
specific language governing permissions and limitations under the License.

@BERI_LICENSE_HEADER_END@

Simulational Model for CHERI MAC (SMSC9115)
===========================================

SMSC9115 chip is supported by U-boot and was simple enough to get emulated
in software. This was the major reason of choosing this MAC chip.

Implementation of code is present in following files:

* EtherCAP.bsv

  Bluespec-based testbench

* ethercap.c
  ethercap.h

  Software implementation of simulational model.


Additional notes
================


CPU bus architecture in CHERI
-----------------------------

CPU is present on the left side.
It is the main MIPSTop module.
Module uses internal busses to fetch instruction and data.
INST and DATA busses are 64-bit each.
These data buses are not related to FPGA manufacturer chip interconnect bus.


      _INST---[Avalon2ClientServer64]--->+--------------+   +-+   
    /64 bits                             |              |   | |  
CPU                                      |Avalon256Merge+-->| | -->BUS SWITCH
    \64 bits                             |              | ^ | |
     --DATA---[Avalon2CLientServer64]--->+--------------+ | +-+   
                                                          |
                                                      256 bits       
                                             [Avalon2ClientServer256]
                                                                     
 <------------------------ SIMULATION ------------------------->
 <-----------------------  SYNTHESIS  -------------------------------------->

Avalon2ClientServer64 converts internal busses to requests understandable by Avalon bus.
Avalon256Merge merges two 64-bit busses together.
The reason for merging is directly related to the system performace.
Later in the system, DRAM memory controller is able to fetch 256-bit chunk of memory at once.
In order not to loose speed, we fetch the whole chunk from DRAM to the CPU cache.
In other words, instead of bringing 1 32-bit value, we bring 8x32bit data.

In the current architecture CHERI Project supports two modes of operation:
simulation and synthesis

In the Bluesim CHERI simulation we only emulate CHERI CPU itself.
The reason for this is that we don't have chip interconnect switch modelled in Bluespec.
This means we're not able to simulate the whole CHERI SoC nowadays, as there
are elements like synthesizable UART or synthesizable bus switch, which aren't
written in Bluespec, and are provided by the FPGA manufacturer.
In the simulation, only CHERI chip is simulated, and the whole environment is mimicked within
CPU's code itself to provide convenient debugging facilities.
This means that in the current implementation, UART is present in the CHERI chip itself.

For synthesis Bluespec code gets translated to Verilog.
Resulting group of files consists code for the MIPSTop - the top-level module of the CPU.
CPU can't exist alone.
It must have peripherials visible for its data and instruction busses.
Most notably CPU needs to store startup instructions and data somewhere (BRAM).
For bigger data capacities DRAM is required, thus DDRAM controller is necesssary.
In order to communicate with the outside world, UART can be used as a form of User Interface.
These elements are synthesizable and are manufacturer specific.
Due to the fact that CHERI's code must be synthesized in commercial IDE specific to FPGA chip
maker, CHERI code must be instantiated in the FPGA environment.

Most notable files necessary for synthesizing CHERI for the Altera environment include:

../../boards/terasic_de4_sopcbuilder/DE4_DDR2.cdf
	Configuration for the quartus_pgm - command line backed for the Altera FPGA programming utility.

../../boards/terasic_de4_sopcbuilder/DE4_DDR2.qsf
	Pin assignment specification.

../../boards/terasic_de4_sopcbuilder/DE4_DDR2.sdc
	Timing constraints configuration.

../../boards/terasic_de4_sopcbuilder/DE4_DDR2.tcl
	Automation script for the synthesis process.
